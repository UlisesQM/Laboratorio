\documentclass[10pt,a4paper]{article}
\usepackage[utf8]{inputenc}
\usepackage[spanish]{babel}
\usepackage{amsmath}
\usepackage{amsfonts}
\usepackage{amssymb}
\usepackage{graphicx}
\usepackage{multicol}
\usepackage[spanish,es-tabla]{babel}
\usepackage{titling}
\usepackage{titlesec}
\usepackage{array}
\usepackage{bm}
\usepackage{afterpage}
\usepackage{float}
\usepackage{graphicx}
\usepackage{epstopdf}
\usepackage{longtable}
\usepackage{xcolor}
\usepackage{epigraph}
\usepackage{float}
\usepackage[backend=biber, style=apa]{biblatex}
\setlength\epigraphwidth{1.5\textwidth}
\addbibresource{Referencias.bib}
\usepackage{subfigure}
\usepackage{anyfontsize}
\usepackage[left=2cm,right=2cm,top=2cm,bottom=2cm]{geometry}
\usepackage[colorlinks=true,
            linkcolor=blue,
            citecolor=blue,
            urlcolor=blue]{hyperref}

\begin{document}
\author{Estudiantes: Ulises Quistial \\ Docente: Msc. Jaime Jaramillo  \\ Técnico de laboratorio: Msc. Alejandra Pinto Erazo}
\title{REGLAS DE LABORATORIO - CIRCUITOS ELÉCTRICOS 
TRABAJO N° 1 }
\date{11 de Octubre de 2024}
\maketitle
\section*{Reglas de Laboratorios}
\begin{itemize}
    \item Uso de equipo de protección personal
    \item Conocer la ubicación de equipos de seguridad
    \item No comer ni beber en el laboratorio
    \item Sigue los procedimientos correctamente
    \item Manejo adecuado de inmediato
    \item Mantener el área de trabajo limpio
    \item Pedir permiso para el uso de la instrumentación
    \item Revisar los equipos antes de usarlos
    \item Respeta las zonas de trabajo
    \item Mantener los cables y conexiones organizados
    \item No dejar experimentos desatendidos
    \item Guarda los equipos después de utilizarlos
    \item No forzar el uso de equipos
    \item No desordenar el área de trabajo con objetos innecesarios
\end{itemize}
\end{document}
